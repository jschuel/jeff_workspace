%%%%%%%%%%%%%%%%%%%%%%%%%%%%%% -*- Mode: Latex -*- %%%%%%%%%%%%%%%%%%%%%%%%%%%%
%% uhtest-body.tex -- 
%% Author          : Robert Brewer
%% Created On      : Fri Oct  2 16:30:37 1998
%% Last Modified By: Robert Brewer
%% Last Modified On: Mon Oct  5 16:01:29 1998
%% RCS: $Id: uhtest-body.tex,v 1.1 1998/10/06 02:07:14 rbrewer Exp $
%%%%%%%%%%%%%%%%%%%%%%%%%%%%%%%%%%%%%%%%%%%%%%%%%%%%%%%%%%%%%%%%%%%%%%%%%%%%%%%
%%   Copyright (C) 1998 Robert Brewer
%%%%%%%%%%%%%%%%%%%%%%%%%%%%%%%%%%%%%%%%%%%%%%%%%%%%%%%%%%%%%%%%%%%%%%%%%%%%%%%
%% 

\chapter{Introduction}



\chapter{Modeling decoherence}

In this section, we follow the work of refs \cite{ALOK, Naikoo} to come up with a parametrization of the mixing induced flavor asymmetry parameter, $\mathcal{A}_\text{mix}(\Delta t)$, that includes contributions of decoherence.

\section{An open $B\bar{B}$ system}

\subsection{Kraus' Theorem}

The notion of closed systems with a single observer is not a realistic representation of the conditions present in $B$ factory experiments. In order to properly model the time evolution of $B$ meson pairs, we must then treat the $B\bar{B}$ system as an \textit{open} system and allow it to interact with its surroundings. These surroundings may include [write more here]

Kraus representations \cite{Kraus} are a convenient way of modeling open system dynamics. The general idea is as follows: Consider a Hilbert space $\mathcal{H}$ composed of two subsystems $\mathcal{H}_a$ and $\mathcal{H}_b$ such that $\mathcal{H}=\mathcal{H}_a\otimes \mathcal{H}_b$. If at a given time, $t$, we represent quantum states by density matrices $\rho(t)$, $\rho_a(t)$, and $\rho_b(t)$, respectively for $\mathcal{H}$, $\mathcal{H}_a$, and $\mathcal{H}_b$, then $\rho_{a}(t)$ ($\rho_b(t)$) is related to $\rho(t)$ by a partial trace over $b$ ($a$), that is,
\begin{align}
\rho_a(t) &= \Tr_b(\rho(t))\nonumber \\
\label{eq:partialtrace}
\rho_b(t) &= \Tr_a(\rho(t)).
\end{align}
Now, since $\mathcal{H}$ is unitary, $\rho(t)$ is simply a unitary transformation of $\rho(0)$:
\begin{align}
\label{eq:unitary}
\rho(t) = U(t)\rho(0)U^\dagger(t),
\end{align} 
for some unitary operator $U(t)$. Plugging \ref{eq:partialtrace} into \ref{eq:unitary} gives us the time evolution of states in $\mathcal{H}_a$
\begin{align}
\rho_a(t) = \Tr_b\left(U(t)\rho(0)U^\dagger(t)\right).
\end{align}
Kraus' theorem states that if $\rho_a(t)$ can be also be written as
\begin{align}
\rho_a(t) = \sum_iK_i(t)\rho(0)K_i^\dagger(t),
\end{align}
with
\begin{align}
\sum_iK_i(t)K_i^\dagger(t)=\mathds{1},
\end{align}
then $K_i(t)$ is a Kraus operator, and $\rho_a(t)$ is completely positive and has a Kraus representation.

\subsection{Open system $B\bar{B}$ dynamics}
To consider the effects of decoherence on the time evolution of neutral $B\bar{B}$ pairs, we start with an orthonormal basis of states
\begin{align}
\ket{B^0}\doteq\begin{pmatrix} 1\\0\\0 \end{pmatrix},\quad \ket{\bar{B}^0}\doteq\begin{pmatrix} 0\\1\\0 \end{pmatrix},\quad \ket{0}\doteq\begin{pmatrix} 0\\0\\1 \end{pmatrix}.
\end{align}
Here, $\ket{B^0}$ ($\ket{\bar{B}^0}$) represent flavor eigenstates for neutral (anti) $B$-mesons, and $\ket{0}$ is the vacuum state used to describe decays. In this basis, initial meson flavor states are written as $\rho_{B^0}(0)=\ket{B^0}\bra{B^0}$ and $\rho_{\bar{B}^0}(0)=\ket{\bar{B}^0}\bra{\bar{B}^0}$. The time evolution of $\rho_{B^0}(0)$ and $\rho_{\bar{B}^0}(0)$ are governed by Kraus operators $\{K_i(t)\}_{i=0}^{5}$ that encode decoherence. In particular
\begin{align}
\label{eq:timeevolution}
\rho_{B^0,\bar{B}^0}(t) = \sum_{i=0}^5K_i(t)\rho^{(i)}_{B^0,\bar{B}^0}(0)K_i^\dagger(t),
\end{align}
with
\begin{align}
\label{eq:firstkraus}
K_0 &= \ket{0}\bra{0} \\
K_1 &= \mathcal{K}_{1+}\left(\ket{B^0}\bra{B^0} + \ket{\bar{B}^0}\bra{\bar{B}^0}\right) +\mathcal{K}_{1-}\left(\frac{p}{q}\ket{B^0}\bra{\bar{B}^0}+\frac{q}{p} \ket{\bar{B}^0}\bra{B^0}\right) \\
K_2 &= \mathcal{K}_2\left(\frac{p+q}{2p}\ket{0}\bra{\bar{B}^0}+\frac{p+q}{2q}\ket{0}\bra{\bar{B}^0}\right) \\
K_3 &= \mathcal{K}_{3+}\frac{p+q}{2p}\ket{0}\bra{\bar{B}^0}+\mathcal{K}_{3-}\frac{p+q}{2q}\ket{0}\bra{\bar{B}^0} \\
K_4 &= \mathcal{K}_{4}\left(\ket{B^0}\bra{B^0} + \ket{\bar{B}^0}\bra{\bar{B}^0} +\frac{p}{q}\ket{B^0}\bra{\bar{B}^0}+\frac{q}{p} \ket{\bar{B}^0}\bra{B^0}\right) \\
\label{eq:lastkraus}
K_5 &= \mathcal{K}_{5}\left(\ket{B^0}\bra{B^0} + \ket{\bar{B}^0}\bra{\bar{B}^0} -\frac{p}{q}\ket{B^0}\bra{\bar{B}^0}-\frac{q}{p} \ket{\bar{B}^0}\bra{B^0}\right).
\end{align}
[Define $p$ and $q$ here]. Substituting equations \ref{eq:firstkraus}--\ref{eq:lastkraus} into \ref{eq:timeevolution}, we find
\begin{align}
\label{eq:firstterm}
\rho_{B^0}^{(0)}(t) &= \ket{0}\braket{0}{B^0}\braket{B^0}{0}\bra{0} = \mathbf{0} \\
\rho_{B^0}^{(1)}(t) &= \left(\mathcal{K}_{1+}\left(\ket{B^0}\bra{B^0} + \ket{\bar{B}^0}\bra{\bar{B}^0}\right) +\mathcal{K}_{1-}\left(\frac{p}{q}\ket{B^0}\bra{\bar{B}^0}+\frac{q}{p} \ket{\bar{B}^0}\bra{B^0}\right)\right)\ket{B^0}\bra{B^0}\nonumber\\
& \times \mathcal{K}^*_{1+}\left(\ket{B^0}\bra{B^0} + \ket{\bar{B}^0}\bra{\bar{B}^0}\right) +\mathcal{K}^*_{1-}\left(\left(\frac{p}{q}\right)^*\ket{\bar{B}^0}\bra{B^0}+\left(\frac{q}{p}\right)^* \ket{B^0}\bra{\bar{B}^0}\right) \nonumber\\
&= \left(\mathcal{K}_{1+}\ket{B^0}\bra{B^0}+\mathcal{K}_{1-}\frac{q}{p}\ket{\bar{B}^0}\bra{B^0}\right)\nonumber \\
& \times \left(\mathcal{K}_{1+}^*\left(\ket{B^0}\bra{B^0}+\ket{\bar{B}^0}\bra{\bar{B}^0}\right)+\mathcal{K}_{1-}^*\left(\left(\frac{p}{q}\right)^*\ket{\bar{B}^0}\bra{B^0}+\left(\frac{q}{p}\right)^*\ket{B^0}\bra{\bar{B}^0}\right)\right)\nonumber\\
&=|\mathcal{K}_{1+}|^2\ket{B^0}\bra{B^0}+\mathcal{K}_{1+}\mathcal{K}_{1-}^*\left(\frac{q}{p}\right)^*\ket{B^0}\bra{\bar{B}^0}\nonumber\\
&+\mathcal{K}_{1-}\mathcal{K}_{1+}^*\frac{q}{p}\ket{\bar{B}^0}\bra{B^0}+|\mathcal{K}_{1-}|^2\left|\frac{q}{p}\right|^2\ket{\bar{B}^0}\bra{\bar{B}^0}\\
\rho_{B^0}^{(2)}(t) &= \mathcal{K}_2\left(\frac{p+q}{2p}\ket{0}\bra{B^0}+\frac{p+q}{2q}\ket{0}\bra{\bar{B}^0}\right)\ket{B^0}\bra{B^0}\nonumber \\
&\times \mathcal{K}_2^*\left(\left(\frac{p+q}{2p}\right)^*\ket{B^0}\bra{0}+\left(\frac{p+q}{2q}\right)^*\ket{\bar{B}^0}\bra{0}\right) \nonumber \\
&= |\mathcal{K}_2|^2\left|\frac{p+q}{2p}\right|^2\ket{0}\bra{0} \\
\rho_{B^0}^{(3)}(t) &= \left(\mathcal{K}_{3+}\frac{p+q}{2p}\ket{0}\bra{B^0}+\mathcal{K}_{3-}\frac{p+q}{2q}\ket{0}\bra{\bar{B}^0}\right)\ket{B^0}\bra{B^0}\nonumber \\
&\times \left(\mathcal{K}_{3+}^*\left(\frac{p+q}{2p}\right)^*\ket{B^0}\bra{0}+\mathcal{K}_{3-}^*\left(\frac{p+q}{2q}\right)^*\ket{\bar{B}^0}\bra{0}\right) \nonumber \\
&= |\mathcal{K}_{3+}|^2\left|\frac{p+q}{2p}\right|^2\ket{0}\bra{0} \\
\rho_{B^0}^{(4)}(t) &= \mathcal{K}_{4}\left(\ket{B^0}\bra{B^0} + \ket{\bar{B}^0}\bra{\bar{B}^0} + \frac{p}{q}\ket{B^0}\bra{\bar{B}^0}+\frac{q}{p} \ket{\bar{B}^0}\bra{B^0}\right)\ket{B^0}\bra{B^0}\nonumber\\
& \times \mathcal{K}^*_{4}\left(\ket{B^0}\bra{B^0} + \ket{\bar{B}^0}\bra{\bar{B}^0} + \left(\frac{p}{q}\right)^*\ket{\bar{B}^0}\bra{B^0}+\left(\frac{q}{p}\right)^* \ket{B^0}\bra{\bar{B}^0}\right) \nonumber\\
&= \mathcal{K}_{4}\left(\ket{B^0}\bra{B^0}+\frac{q}{p}\ket{\bar{B}^0}\bra{B^0}\right)\nonumber \\
& \times \mathcal{K}_{4}^*\left(\ket{B^0}\bra{B^0}+\ket{\bar{B}^0}\bra{\bar{B}^0}+\left(\frac{p}{q}\right)^*\ket{\bar{B}^0}\bra{B^0}+\left(\frac{q}{p}\right)^*\ket{B^0}\bra{\bar{B}^0}\right)\nonumber\\
&=|\mathcal{K}_{4}|^2\ket{B^0}\bra{B^0}+|\mathcal{K}_{4}|^2\left(\frac{q}{p}\right)^*\ket{B^0}\bra{\bar{B}^0} +|\mathcal{K}_{4}|^2\frac{q}{p}\ket{\bar{B}^0}\bra{B^0}+|\mathcal{K}_{4}|^2\left|\frac{q}{p}\right|^2\ket{\bar{B}^0}\bra{\bar{B}^0}
\end{align}
\begin{align}
\rho_{B^0}^{(5)}(t) &= \mathcal{K}_{5}\left(\ket{B^0}\bra{B^0} + \ket{\bar{B}^0}\bra{\bar{B}^0} - \frac{p}{q}\ket{B^0}\bra{\bar{B}^0}-\frac{q}{p} \ket{\bar{B}^0}\bra{B^0}\right)\ket{B^0}\bra{B^0}\nonumber\\
& \times \mathcal{K}^*_{5}\left(\ket{B^0}\bra{B^0} + \ket{\bar{B}^0}\bra{\bar{B}^0} - \left(\frac{p}{q}\right)^*\ket{\bar{B}^0}\bra{B^0}-\left(\frac{q}{p}\right)^* \ket{B^0}\bra{\bar{B}^0}\right) \nonumber\\
&= \mathcal{K}_{5}\left(\ket{B^0}\bra{B^0}-\frac{q}{p}\ket{\bar{B}^0}\bra{B^0}\right)\nonumber \\
& \times \mathcal{K}_{5}^*\left(\ket{B^0}\bra{B^0}+\ket{\bar{B}^0}\bra{\bar{B}^0}-\left(\frac{p}{q}\right)^*\ket{\bar{B}^0}\bra{B^0}-\left(\frac{q}{p}\right)^*\ket{B^0}\bra{\bar{B}^0}\right)\nonumber\\
\label{eq:lastterm}
&=|\mathcal{K}_{5}|^2\ket{B^0}\bra{B^0}-|\mathcal{K}_{5}|^2\left(\frac{q}{p}\right)^*\ket{B^0}\bra{\bar{B}^0} -|\mathcal{K}_{5}|^2\frac{q}{p}\ket{\bar{B}^0}\bra{B^0}+|\mathcal{K}_{5}|^2\left|\frac{q}{p}\right|^2\ket{\bar{B}^0}\bra{\bar{B}^0}
\end{align}
Summing everything up in equations \ref{eq:firstterm}-\ref{eq:lastterm} and writing in matrix form, we find
\begin{align}
\label{eq:bstate}
\rho_{B^0}(t) = \begin{pmatrix} |\mathcal{K}_{1+}|^2 + |\mathcal{K}_4|^2 + |\mathcal{K}_5|^2 & \left(\frac{q}{p}\right)^*\left(\mathcal{K}_{1+}\mathcal{K}_{1-}^*+|\mathcal{K}_4|^2-|\mathcal{K}_5|^2\right) & 0 \\
\left(\frac{q}{p}\right)\left(\mathcal{K}_{1+}^*\mathcal{K}_{1-}+|\mathcal{K}_4|^2-|\mathcal{K}_5|^2\right) & \left|\frac{q}{p}\right|^2\left(|\mathcal{K}_{1-}|^2 + |\mathcal{K}_4|^2 + |\mathcal{K}_5|^2 \right) & 0 \\
0 & 0 & \left|\frac{p+q}{2p}\right|^2(|\mathcal{K}_2|^2 + |\mathcal{K}_3|^2) \end{pmatrix}.
\end{align}

The $\mathcal{K}$ coefficients in equations \ref{eq:firstkraus}-\ref{eq:bstate} are defined as follows \cite{Naikoo}:
\begin{align}
\mathcal{K}_{1\pm} &= \frac{1}{2}\left(e^{-(2im_L+\Gamma_L+\lambda)t/2}\pm e^{-(2im_H+\Gamma_H+\lambda)t/2}\right) \nonumber \\
\mathcal{K}_2 &= \sqrt{\frac{\Re\left[\frac{p-q}{p+q}\right]}{|p|^2-|q|^2}\left(1-e^{-\Gamma_L t}-\frac{(|p|^2-|q|^2)^2 |1-e^{-(\Gamma+\lambda-i\Delta m)t}|^2}{1-e^{-\Gamma_H t}}\right)} \nonumber \\
\mathcal{K}_{3\pm} &= \sqrt{\frac{\Re\left[\frac{p-q}{p+q}\right]}{(|p|^2-|q|^2)(1-e^{-\Gamma_H t})}}\left[1-e^{-\Gamma_H t}\pm(1-e^{-(\Gamma + \lambda-i\Delta m)t})(|p|^2-|q|^2)\right] \nonumber \\
\mathcal{K}_4 &= \frac{e^{-\Gamma_L t/2}}{2}\sqrt{1-e^{-\lambda t}} \nonumber \\
\mathcal{K}_5 &= \frac{e^{-\Gamma_H t/2}}{2}\sqrt{1-e^{-\lambda t}},
\end{align}
where $\lambda$ is a decoherence parameter, $\Gamma = \frac{1}{2}(\Gamma_L+\Gamma_H)$ is the decay width, $\Delta m = m_H-m_L$ is the mass difference between the $B$ mass eigenstates $\ket{B^0_L}$ and $\ket{B^0_H}$, related to the $B$ flavor eigenstates by
\begin{align}
\ket{B^0_L}=p\ket{B^0}+q\ket{\bar{B}^0}, \quad \ket{B_H^0}=p\ket{B^0}-q\ket{\bar{B}^0},
\end{align}
with $|p|^2 + |q|^2 = 1$. Substituting the expressions for $\mathcal{K}_{1\pm}$, $\mathcal{K}_4$ and $\mathcal{K}_5$\footnote{We leave $\mathcal{K}_{2,3\pm}$ as is because as we'll soon see, they don't have an effect on our calculations of interest.} into \ref{eq:bstate}, we first find:
\begin{align}
\rho_{B^0}(t)_{00} &= |\mathcal{K}_{1+}|^2 + |\mathcal{K}_4|^2 + |\mathcal{K}_5|^2 \nonumber \\
&= \frac{1}{4}\left(e^{-(2im_L+\Gamma_L +\lambda)t/2}+e^{-(2im_H+\Gamma_H +\lambda)t/2}\right)\left(e^{-(-2im_L+\Gamma_L +\lambda)t/2}+e^{-(-2im_H+\Gamma_H +\lambda)t/2}\right) \nonumber \\
&+ \left(\frac{1-e^{-\lambda t}}{4}\right)(e^{-\Gamma_L t} + e^{-\Gamma_H t}) \nonumber \\
&= \frac{1}{4}\left(e^{-(\Gamma_L + \lambda)t} + e^{-(\Gamma_H + \lambda)t} + e^{-(\Gamma+\lambda)t}\left(e^{i\Delta m t}+e^{-i\Delta m t}\right)\right)+ \left(\frac{1-e^{-\lambda t}}{4}\right)(e^{-\Gamma_L t} + e^{-\Gamma_H t}) \nonumber \\
&= \frac{1}{4}(e^{-\Gamma_L t} + e^{-\Gamma_H t})+\frac{e^{-(\Gamma + \lambda) t}}{2}\cos(\Delta m t) \nonumber \\
&= \frac{e^{\Delta \Gamma t}+1}{4e^{\Gamma_L t}} + \frac{e^{-(\Gamma + \lambda) t}}{2}\cos(\Delta m t) \nonumber \\
&= \frac{e^{\frac{\Delta \Gamma t}{2}}+e^{-\frac{\Delta \Gamma t}{2}}}{4e^{\Gamma_L t}e^{-\frac{\Delta \Gamma t}{2}}} + \frac{e^{-(\Gamma + \lambda) t}}{2}\cos(\Delta m t) \nonumber \\
&= \frac{\cosh\left(\frac{\Delta \Gamma t}{2}\right)}{2e^{(\Gamma_L+\Gamma_H)t/2}} + \frac{e^{-(\Gamma + \lambda) t}}{2}\cos(\Delta m t) \nonumber \\
&= \frac{e^{-\Gamma t}}{2}\left(\cosh\left(\frac{\Delta \Gamma t}{2}\right) + \cos(\Delta m t)e^{-\lambda t}\right) \nonumber \\
&= \frac{e^{-\Gamma t}}{2}\left(a_{ch} + a_c e^{-\lambda t}\right),
\end{align}
where $a_{ch}\equiv \cosh\left(\frac{\Delta \Gamma t}{2}\right)$ and $a_c\equiv \cos(\Delta m t)$. Doing the same for $\rho_{B^0}(t)_{01}$, $\rho_{B^0}(t)_{10}$, and $\rho_{B^0}(t)_{11}$:
\begin{align}
\rho_{B^0}(t)_{01} &= \left(\frac{q}{p}\right)^*(\mathcal{K}_{1+}\mathcal{K}_{1-}^* + |\mathcal{K}_4|^2 - |\mathcal{K}_5|^2) \nonumber \\
&= \frac{1}{4}\left(\frac{q}{p}\right)^*\left(e^{-(2im_L+\Gamma_L +\lambda)t/2}+e^{-(2im_H+\Gamma_H +\lambda)t/2}\right)\left(e^{-(-2im_L+\Gamma_L +\lambda)t/2}-e^{-(-2im_H+\Gamma_H +\lambda)t/2}\right) \nonumber \\
&+ \left(\frac{q}{p}\right)^*\left(\frac{1-e^{-\lambda t}}{4}\right)(e^{-\Gamma_L t} - e^{-\Gamma_H t}) \nonumber \\
&= \frac{1}{4}\left(\frac{q}{p}\right)^*\left(e^{-(\Gamma_L+\lambda)t}-e^{-(\Gamma_H+\lambda)t}+e^{-(\Gamma+\lambda)t}\left( e^{-i\Delta m t}-e^{i\Delta m t}\right)\right) + \left(\frac{q}{p}\right)^*\left(\frac{1-e^{-\lambda t}}{4}\right)(e^{-\Gamma_L t} - e^{-\Gamma_H t}) \nonumber \\
&= \left(\frac{q}{p}\right)^*\frac{e^{-\Gamma_L t}-e^{-\Gamma_H t}}{4}-\left(\frac{q}{p}\right)^*\frac{ie^{-(\Gamma + \lambda)t}}{2}\sin(\Delta m t) \nonumber \\
&= \left(\frac{q}{p}\right)^*\frac{1-e^{\Delta \Gamma t}}{4e^{\Gamma_2 t}}-\left(\frac{q}{p}\right)^*\frac{ie^{-(\Gamma + \lambda)t}}{2}\sin(\Delta m t) \nonumber \\
&= \left(\frac{q}{p}\right)^*\frac{e^{-\frac{\Delta \Gamma t}{2}}-e^{\frac{\Delta \Gamma t}{2}}}{4e^{\Gamma_2 t}e^{-\frac{\Delta \Gamma t}{2}}}-\left(\frac{q}{p}\right)^*\frac{ie^{-(\Gamma + \lambda)t}}{2}\sin(\Delta m t) \nonumber \\
&= \left(\frac{q}{p}\right)^*\frac{-2\sinh\left(\frac{\Delta \Gamma t}{2}\right)}{4e^{\Gamma t}}-\left(\frac{q}{p}\right)^*\frac{ie^{-(\Gamma + \lambda)t}}{2}\sin(\Delta m t) \nonumber \\
&=-\left(\frac{q}{p}\right)^*\frac{e^{-\Gamma t}}{2}\left(a_{sh}+ia_s e^{-\lambda t}\right), \\
\rho_{B^0}(t)_{10} &= \left(\frac{q}{p}\right)(\mathcal{K}_{1+}^*\mathcal{K}_{1-} + |\mathcal{K}_4|^2 - |\mathcal{K}_5|^2) \nonumber \\
&= \frac{1}{4}\left(\frac{q}{p}\right)\left(e^{-(-2im_L+\Gamma_L +\lambda)t/2}+e^{-(-2im_H+\Gamma_H +\lambda)t/2}\right)\left(e^{-(2im_L+\Gamma_L +\lambda)t/2}-e^{-(2im_H+\Gamma_H +\lambda)t/2}\right) \nonumber \\
&+ \left(\frac{q}{p}\right)\left(\frac{1-e^{-\lambda t}}{4}\right)(e^{-\Gamma_L t} - e^{-\Gamma_H t}) \nonumber \\
&= \frac{1}{4}\left(\frac{q}{p}\right)\left(e^{-(\Gamma_L+\lambda)t}-e^{-(\Gamma_H+\lambda)t}+e^{-(\Gamma+\lambda)t}\left( e^{i\Delta m t}-e^{-i\Delta m t}\right)\right) + \left(\frac{q}{p}\right)\left(\frac{1-e^{-\lambda t}}{4}\right)(e^{-\Gamma_L t} - e^{-\Gamma_H t}) \nonumber \\
&= \left(\frac{q}{p}\right)\frac{e^{-\Gamma_L t}-e^{-\Gamma_H t}}{4}+\left(\frac{q}{p}\right)\frac{ie^{-(\Gamma + \lambda)t}}{2}\sin(\Delta m t) \nonumber \\
&= \left(\frac{q}{p}\right)\frac{1-e^{\Delta \Gamma t}}{4e^{\Gamma_2 t}}+\left(\frac{q}{p}\right)\frac{ie^{-(\Gamma + \lambda)t}}{2}\sin(\Delta m t) \nonumber \\
&= \left(\frac{q}{p}\right)\frac{e^{-\frac{\Delta \Gamma t}{2}}-e^{\frac{\Delta \Gamma t}{2}}}{4e^{\Gamma_2 t}e^{-\frac{\Delta \Gamma t}{2}}}+\left(\frac{q}{p}\right)\frac{ie^{-(\Gamma + \lambda)t}}{2}\sin(\Delta m t) \nonumber \\
&= \left(\frac{q}{p}\right)\frac{-2\sinh\left(\frac{\Delta \Gamma t}{2}\right)}{4e^{\Gamma t}}+\left(\frac{q}{p}\right)\frac{ie^{-(\Gamma + \lambda)t}}{2}\sin(\Delta m t) \nonumber \\
&=\left(\frac{q}{p}\right)\frac{e^{-\Gamma t}}{2}\left(-a_{sh}+ia_s e^{-\lambda t}\right),
\end{align}
\begin{align}
\rho_{B^0}(t)_{11} &= \left|\frac{q}{p}\right|^2(|\mathcal{K}_{1-}|^2 + |\mathcal{K}_4|^2 + |\mathcal{K}_5|^2) \nonumber \\
&= \frac{1}{4}\left|\frac{q}{p}\right|^2\left(e^{-(2im_L+\Gamma_L +\lambda)t/2}-e^{-(2im_H+\Gamma_H +\lambda)t/2}\right)\left(e^{-(-2im_L+\Gamma_L +\lambda)t/2}-e^{-(-2im_H+\Gamma_H +\lambda)t/2}\right) \nonumber \\
&+ \left|\frac{q}{p}\right|^2\left(\frac{1-e^{-\lambda t}}{4}\right)(e^{-\Gamma_L t} + e^{-\Gamma_H t}) \nonumber \\
&= \frac{1}{4}\left|\frac{q}{p}\right|^2\left(e^{-(\Gamma_L + \lambda)t} + e^{-(\Gamma_H + \lambda)t} - e^{-(\Gamma+\lambda)t}\left(e^{i\Delta m t}+e^{-i\Delta m t}\right)\right)+ \left|\frac{q}{p}\right|^2\left(\frac{1-e^{-\lambda t}}{4}\right)(e^{-\Gamma_L t} + e^{-\Gamma_H t}) \nonumber \\
&= \frac{1}{4}\left|\frac{q}{p}\right|^2(e^{-\Gamma_L t} + e^{-\Gamma_H t})-\left|\frac{q}{p}\right|^2\frac{e^{-(\Gamma + \lambda) t}}{2}\cos(\Delta m t) \nonumber \\
&= \left|\frac{q}{p}\right|^2\frac{e^{\Delta \Gamma t}+1}{4e^{\Gamma_L t}} -\left|\frac{q}{p}\right|^2 \frac{e^{-(\Gamma + \lambda) t}}{2}\cos(\Delta m t) \nonumber \\
&= \left|\frac{q}{p}\right|^2\frac{e^{\frac{\Delta \Gamma t}{2}}+e^{-\frac{\Delta \Gamma t}{2}}}{4e^{\Gamma_L t}e^{-\frac{\Delta \Gamma t}{2}}} -\left|\frac{q}{p}\right|^2 \frac{e^{-(\Gamma + \lambda) t}}{2}\cos(\Delta m t) \nonumber \\
&= \left|\frac{q}{p}\right|^2\frac{\cosh\left(\frac{\Delta \Gamma t}{2}\right)}{2e^{(\Gamma_L+\Gamma_H)t/2}} -\left|\frac{q}{p}\right|^2 \frac{e^{-(\Gamma + \lambda) t}}{2}\cos(\Delta m t) \nonumber \\
&= \left|\frac{q}{p}\right|^2\frac{e^{-\Gamma t}}{2}\left(\cosh\left(\frac{\Delta \Gamma t}{2}\right) - \cos(\Delta m t)e^{-\lambda t}\right) \nonumber \\
&= \left|\frac{q}{p}\right|^2\frac{e^{-\Gamma t}}{2}\left(a_{ch} - a_c e^{-\lambda t}\right),
\end{align}
where $a_{sh}\equiv \sinh\left(\frac{\Delta \Gamma t}{2}\right)$ and $a_s\equiv \sin(\Delta m t)$. Plugging this all into \ref{eq:bstate} we find
\begin{align}
\rho_{B^0}(t) = \frac{e^{-\Gamma t}}{2}\begin{pmatrix}
a_{ch} + e^{-\lambda t}a_c & -\left(\frac{q}{p}\right)^*\left(a_{sh}+ ie^{-\lambda t}a_s\right) & 0 \\ 
\left(\frac{q}{p}\right)\left(-a_{sh} + ie^{-\lambda t}a_s \right) & \left|\frac{q}{p}\right|^2 \left(a_{ch}- e^{-\lambda t}a_c\right) & 0 \\
0 & 0 & \rho_{B^0}(t)_{22}
\end{pmatrix}.
\end{align}
We can perform this same procedure and time evolve $\rho_{\bar{B}^0}(0)$ using these same six Kraus operators $\{K_i(t)\}_{i=1}^{6}$ to get $\rho_{\bar{B}^0}(t)$. We summarize the results after doing so as follows

\begin{align}
\label{eq:density}
\rho_{\pm}(t) = \frac{e^{-\Gamma t}}{2}\begin{pmatrix}
a_{ch} \pm e^{-\lambda t}a_c & -\left(\frac{q}{p}\right)^*\left(a_{sh}\pm ie^{-\lambda t}a_s\right) & 0 \\
\left(\frac{q}{p}\right)\left(-a_{sh} \pm ie^{-\lambda t}a_s \right) & \left|\frac{q}{p}\right|^2 \left(a_{ch}\mp e^{-\lambda t}a_c\right)  & 0 \\
0 & 0 & \rho_{\pm}(t)_{22})
\end{pmatrix},
\end{align}
where $\rho_+(t)$ and $\rho_-(t)$ correspond to $B^0$ and $\bar{B}^0$, respectively.

\begin{align}
\label{eq:observable}
\mathcal{O}_f = 
\begin{pmatrix}
|A_f|^2 & A_f^*\bar{A}_f & 0 \\
A_f\bar{A}_f^* & |\bar{A}_f|^2 & 0 \\
0 & 0 & 0
\end{pmatrix},
\end{align}
where $A_f \equiv A(B^0\rightarrow f)$ and $\bar{A}_f \equiv A(\bar{B}^0\rightarrow f)$.

With this construction, the probability of a $B^0$ or $\bar{B}^0$ decaying into state $f$ at time $t$ is computed as
\begin{align}
\label{eq:trace}
P_{f\pm}(t) = \Tr(\mathcal{O}_f\mathcal{\rho_\pm}),
\end{align}
with $P_+$ corresponding to an initial $B^0$ and $P_-$, an initial $\bar{B}^0$. An observable
\begin{align}
\label{eq:CP}
\mathcal{A}_f =\frac{P_-(t)-P_+(t)}{P_-(t)+P_+(t)}
\end{align}
can be defined, and when we set $f$ to correspond to the golden mode, that is, $f = J/\psi K_S$, this observable represents CP violating asymmetry used to determine $\sin(2\phi_1)$ in \cite{Abe1}. To show this, we compute the probabilities in \ref{eq:CP} using \ref{eq:observable}, \ref{eq:density} and \ref{eq:trace}. Doing this, we find
\begin{align}
\label{eq:probs}
P_{J/\psi K_S\pm}(t) &= \frac{1}{2}e^{-\Gamma_d t}\Bigg( \Bigg.|A_f|^2(a_{ch}\pm e^{-\lambda t}a_c)+|\bar{A}_f|^2(a_{ch}\mp e^{-\lambda t}a_c) \nonumber\\  &-A_f^*\bar{A}_f(a_{sh}\mp ie^{-\lambda t}a_s) - A_f\bar{A}_f^*(a_{sh}\pm ie^{-\lambda t}a_s)\Bigg) \Bigg. ;\quad f=J/\psi K_S.
\end{align}
Factoring out $|A_f|^2=A_fA_f^*$ from both the numerator and denominator of \ref{eq:CP} and defining $z\equiv A(\bar{B}^0\rightarrow J/\psi K_S)/A(B^0\rightarrow J/\psi K_S)$, we obtain
\begin{align}
\label{eq:cpviolation}
\mathcal{A}_{J/\Psi K_S}(t,\lambda)&=\frac{2(|z|^2-1)a_c - 2iz a_s + 2iz^*a_s}{2(|z|^2+1)a_{ch} -2z a_{sh} -2z^*a_{sh}}e^{-\lambda t}\nonumber\\
&=\frac{(|z|^2-1)a_c + 2\Im(z)a_s}{(|z|^2+1)a_{ch} -2\Re(z) a_{sh}}e^{-\lambda t}\nonumber\\
&=\frac{(|z|^2-1)\cos(\Delta m_d t) + 2\Im(z)\sin(\Delta m_d t)}{(|z|^2+1)\cosh(\Delta \Gamma_d t/2) -2\Re(z) \sinh(\Delta\Gamma_d t/2)}e^{-\lambda t}.
\end{align}
where we used the fact that the decay amplitudes are, in general, complex numbers, so $\Re(z)=\frac{z+z^*}{2}$ and $\Im(z) = \frac{z-z^*}{2i}$. We see that \ref{eq:cpviolation} is indeed the well-known mixing and decay-induced CP asymmetry expression with $\Im(z)\approx \sin(2\phi_1)$ \cite{Abe1,Abe2}, however it includes an additional \textit{decoherence} term, $e^{-\lambda t}$, where we refer to $\lambda$ as the \textit{decoherence parameter}. We see that in the case of no decoherence ($\lambda = 0$), \ref{eq:cpviolation} is exactly the CP asymmetry expression described above.

In the case where a $B^0$ decays into a state that is inaccessible from a $\bar{B}^0$ decay, it follows that $z=0$ \cite{Bevan}, which will lead us to an expression for the time dependent mixing asymmetry $\mathcal{A}_{\text{mix}}$. Indeed, if we consider \ref{eq:probs} and set the final state to $B^0$ (or $\bar{B}^0$), we can compute flavor mixing probabilities. For example, if we were to compute $P_{B^0\rightarrow \bar{B}^0}(t)$, we set $A_f = 0$ and $\bar{A}_f = 1$ in \ref{eq:probs}, leading us to
\begin{align}
\label{eq:prob1}
P_{B^0\rightarrow \bar{B}^0}(t) \sim \cosh(\Delta \Gamma_d t/2)-e^{-\lambda t}\cos(\Delta m_dt).
\end{align}
Similarly, for the other mixing combinations, we would set $A_f = 0$ and $\bar{A}_f = 1$ for $P_{\bar{B}^0\rightarrow \bar{B}^0}(t)$, and we would set $A_f = 1$ and $\bar{A}_f = 0$ for $P_{\bar{B}^0\rightarrow B^0}(t)$ and $P_{B^0\rightarrow B^0}(t)$, giving
\begin{align}
P_{\bar{B}^0\rightarrow \bar{B}^0}(t) \sim \cosh(\Delta \Gamma_d t/2)+e^{-\lambda t}\cos(\Delta m_dt) \\
P_{B^0\rightarrow B^0}(t) \sim \cosh(\Delta \Gamma_d t/2)+e^{-\lambda t}\cos(\Delta m_dt)\\
\label{eq:prob2}
P_{\bar{B}^0\rightarrow B^0}(t) \sim \cosh(\Delta \Gamma_d t/2)-e^{-\lambda t}\cos(\Delta m_dt).
\end{align}

Now let's consider a $B\bar{B}$ produced from the hadronization of $e^+e^-\rightarrow \Upsilon(4S)\rightarrow b\bar{b}$, which is the mechanism for $B$ production at Belle. Since the $\Upsilon(4S)$ is spin 1, it follows from conservation of angular momentum that the resulting $B\bar{B}$ pair will be in a coherent $P$-wave state, which means that at a certain time $t_0$, nominally the decay time of the first $B^0$ in the $B\bar{B}$ pair, the flavor the decaying $B$ \textit{must} be the opposite of the flavor of the other $B$. This means the probability of observing opposite flavor $P_{B^0\bar{B}^0\rightarrow B^0\bar{B}^0}$ or same flavor pairs $P_{B^0\bar{B}^0\rightarrow B^0B^0 \text{ or } \bar{B}^0\bar{B}^0}$ is determined from the proper time difference between the decays of the two $B$'s, $\Delta t\equiv t_1-t_0$. With this knowledge at our disposal, we see from equations \ref{eq:prob1}--\ref{eq:prob2} that \textit{mixing} (creation of same flavor pair) is the result of second $B$ changing flavor and thus has a minus sign in its oscillation probability, whereas an unmixed (opposite flavor) pair has a plus sign in its oscillation probability. This means we can write
\begin{align}
P_{B^{0}\bar{B}^0\rightarrow B^{0}\bar{B}^0}(\Delta t) \sim \cosh(\Delta \Gamma_d \Delta t/2)+e^{-\lambda \Delta t}\cos(\Delta m_d\Delta t) = P_+(\Delta t)\\
P_{B^{0}\bar{B}^0\rightarrow B^{0}B^0\text{ or } \bar{B}^0\bar{B}^0}(\Delta t) \sim \cosh(\Delta \Gamma_d \Delta t/2)-e^{-\lambda \Delta t}\cos(\Delta m_d\Delta t) = P_-(\Delta t).
\end{align}
Finally, we now define the time dependent mixing asymmetry, $\mathcal{A}_\text{mix}(\Delta t)$ as
\begin{align}
\mathcal{A}_\text{mix}(\Delta t)\equiv \frac{P_+(\Delta t)-P_-(\Delta t)}{P_+(\Delta t)+P_-(\Delta t)} = \frac{\cos(\Delta m_d\Delta t)}{\cosh(\Delta \Gamma_d \Delta t/2)}e^{-\lambda\Delta t}.
\end{align}
Just like with \ref{eq:cpviolation}, we see that we now have an expression for the time-dependent asymmetry which also manifestly depends on decoherence parameter $\lambda$.

Note: Make sure to cite \cite{Bertlmann} for the construction of the decoherence parameter in the Kaon system, \cite{Banerjee} for the explicit treatment of $B$ mesons as an open system, \cite{Benatti} and \cite{Dixit} and \cite{ALOK} for their explicit usage of Go's dataset to show how updates using Belle I data can still be helpful. And of course, cite Go \cite{Go}. Also, also, it looks like \cite{Naikoo} explicitly constructs $\rho(t)$ and $\mathcal{O}$ that I use in this chapter. \cite{Lindblad} and \cite{Gorini} when bringing up dynamical positive semigroup formalism. 
\section{Bibliography Citations}
Citing references to your bibliography is easy \cite{belle}
\cite{kekb}. First you build a BibTeX file which contains the
records for all of the works you wish to cite. This file ends with a ``{\tt
.bib}'' extension. Then in your body you use the ``{\tt $\backslash$cite}''
command with the label you gave to the record in question. The final steps are: 
run LaTeX once, run BibTeX, and then run LaTeX twice more. You should now have
a bibliography that includes those citations.

\chapter{Conclusion}

This is going to be the chapter where I check the length of the page to make
sure the bottom margin works out all right.  I hope you don't mind long
annoying and useless paragraphs because you are sure to get a lot of them here!

\section{Widgets}

This is going to be the chapter where I check the length of the page
to make sure the bottom margin works out all right.  I hope you don't
mind long annoying and useless paragraphs because you are sure to get
a lot of them here!

\subsection{Sub-Widgets}

This is going to be the chapter where I check the length of the page
to make sure the bottom margin works out all right.  I hope you don't
mind long annoying and useless paragraphs because you are sure to get
a lot of them here!

\subsubsection{Sub-Sub-Widgets}

This is going to be the chapter where I check the length of the page
to make sure the bottom margin works out all right.  I hope you don't
mind long annoying and useless paragraphs because you are sure to get
a lot of them here!

\paragraph{Para-Widgets}

This is going to be the chapter where I check the length of the page
to make sure the bottom margin works out all right.  I hope you don't
mind long annoying and useless paragraphs because you are sure to get
a lot of them here!

\subparagraph{Sub-Para-Widgets}

This is going to be the chapter where I check the length of the page
to make sure the bottom margin works out all right.  I hope you don't
mind long annoying and useless paragraphs because you are sure to get
a lot of them here!

This is going to be the chapter where I check the length of the page
to make sure the bottom margin works out all right.  I hope you don't
mind long annoying and useless paragraphs because you are sure to get
a lot of them here!

This is going to be the chapter where I check the length of the page
to make sure the bottom margin works out all right.  I hope you don't
mind long annoying and useless paragraphs because you are sure to get
a lot of them here!

This is going to be the chapter where I check the length of the page
to make sure the bottom margin works out all right.  I hope you don't
mind long annoying and useless paragraphs because you are sure to get
a lot of them here!

This is going to be the chapter where I check the length of the page
to make sure the bottom margin works out all right.  I hope you don't
mind long annoying and useless paragraphs because you are sure to get
a lot of them here!

This is going to be the chapter where I check the length of the page
to make sure the bottom margin works out all right.  I hope you don't
mind long annoying and useless paragraphs because you are sure to get
a lot of them here!

This is going to be the chapter where I check the length of the page
to make sure the bottom margin works out all right.  I hope you don't
mind long annoying and useless paragraphs because you are sure to get
a lot of them here!

This is going to be the chapter where I check the length of the page
to make sure the bottom margin works out all right.  I hope you don't
mind long annoying and useless paragraphs because you are sure to get
a lot of them here!

This is going to be the chapter where I check the length of the page
to make sure the bottom margin works out all right.  I hope you don't
mind long annoying and useless paragraphs because you are sure to get
a lot of them here!

This is going to be the chapter where I check the length of the page
to make sure the bottom margin works out all right.  I hope you don't
mind long annoying and useless paragraphs because you are sure to get
a lot of them here!

This is going to be the chapter where I check the length of the page
to make sure the bottom margin works out all right.  I hope you don't
mind long annoying and useless paragraphs because you are sure to get
a lot of them here!

This is going to be the chapter where I check the length of the page
to make sure the bottom margin works out all right.  I hope you don't
mind long annoying and useless paragraphs because you are sure to get
a lot of them here!

This is going to be the chapter where I check the length of the page
to make sure the bottom margin works out all right.  I hope you don't
mind long annoying and useless paragraphs because you are sure to get
a lot of them here!

This is going to be the chapter where I check the length of the page
to make sure the bottom margin works out all right.  I hope you don't
mind long annoying and useless paragraphs because you are sure to get
a lot of them here!

This is going to be the chapter where I check the length of the page
to make sure the bottom margin works out all right.  I hope you don't
mind long annoying and useless paragraphs because you are sure to get
a lot of them here!

This is going to be the chapter where I check the length of the page
to make sure the bottom margin works out all right.  I hope you don't
mind long annoying and useless paragraphs because you are sure to get
a lot of them here!
